\documentclass[UTF8]{article}

\usepackage[UTF8]{ctex}
\usepackage{graphicx}
\usepackage{amsmath}

\title{我的Linux工作环境介绍}

\author{鲁衍坤 \\ 信计2101 3210105173}
\date{June 29, 2022}

\bibliographystyle{plain}

\begin{document}

\maketitle

\section{Linux发行版信息}

在shell中调用\verb|lsb_release -a|后,显示结果为:\cite{ct1}

\begin{verbatim}
Distributor ID:	Ubuntu
Description:	Ubuntu 22.04 LTS
Release:	22.04
Codename:	jammy
\end{verbatim}

意味着本机发行版名称为Ubuntu 22.04 LTS,版本号为22.04。

\section{个性化调整}

\subsection{安装的软件}

笔者安装的软件很少,主要为以下所列:

\begin{itemize}
\item 浏览器FireFox
\item 文本编辑器emacs
\item c和c++语言的编译器gcc和g++
\item 文档阅读器okular
\item 代码管理系统git
\end{itemize}

\subsection{额外配置}

笔者同时对本机做了一些额外配置,具体如下:

将本地文本的标准格式改为UTF-8,以便于使用中文和其他文字。\\
并且在用户目录下的 .bash 文件中添加了\verb|alias emacs='LC_CTYPE=zh_CN.utf-8 emacs'|来保证emacs运行时也使用UTF-8的文本格式。

笔者的机器同样修改了emacs的配置文件.emacs,具体配置如下:

\begin{verbatim}
(global-font-lock-mode t)
(shown-paren-mode t)

'(colum-number-mode t)
'(display-time-mode t)

(global-set-key(kbd "TAB") 'self-insert-command)
(setqdefault-tab-width 4)
(setqbackward-delete-char-untabify-method nil)
(defunmy-c-mode-hook ()
  (setq c-basic-offset 4
        c-indent-level 4
        c-default-style "bsd"))
(add-hook'c-mode-common-hook 'my-c-mode-hook)
\end{verbatim}

它们的作用为:

\begin{itemize}
  \item 语法高亮
  \item 括号匹配高亮
  \item 显示行数
  \item 显示时间
  \item 将键盘键入的“Tab”改为输入4个空格
  \begin{itemize}
    \item 这项修改不会将Tab键入的结果直接替换为四个空格,只是显示如此
    \item 这项修改还支持了Tab键入的空白内容只需要一次delete即可删除(而非4次)
    \item c语言的智能缩进没有受到影响,也不会影响这项修改
  \end{itemize}
\end{itemize}

\subsection{下一步的工作}

\subsubsection{使用场合}

笔者是一名苦比大一生,所以接下来的大二秋冬学期所将要修读的课程《数据结构与算法》中应该会用到Linux环境。尤其是编写高级数据结构处理大数据时(如对$10^5$个数建立可持久化线段树),需要一个高效的运算环境,而Linux很适合做这样的工作。\\
并且,在今后练习和使用c语言(或c++)时很有可能继续使用Linux系统,因为这个系统相比于windows系统更加高效且不易受影响。

\subsubsection{未来需求}

笔者认为本机的配置已经可以满足未来的学习需求(至少在大二这一年内如此):能够写代码/写文章/浏览网页,并且管理和修改都相当高效。

\subsection{保证文件安全}

首先,我在本地建立了一个git仓库用于管理我所写的文件。

比如笔者写下这句话时我的本地git仓库的状态是这样的:

\begin{verbatim}
$ git status
位于分支 master
您的分支领先 'origin/master' 共 1 个提交。
  (使用 "git push" 来发布您的本地提交)
$ git log
commit 12ccff8a3f2345bd83f36f60e7b137f0bfe3429e (HEAD -> master)
Author: FallCloud <214467369@qq.com>
Date:   Wed Jun 29 13:37:49 2022 +0800

    add homework3

commit fa3abfebd4e639791b4f51db42dbb41bbd30fc1d (origin/master)
Author: FallCloud <214467369@qq.com>
Date:   Tue Jun 28 16:12:35 2022 +0800

    complete second homework
//只截取了最近两次修改
\end{verbatim}

同时,我也在github上建立了我的远程仓库。从上面的代码可以看到笔者写下这句话时远程仓库的版本是本次修改前的版本。

\bibliography{quote}

\end{document}
