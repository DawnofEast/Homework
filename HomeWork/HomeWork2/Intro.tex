\documentclass[UTF8]{article}

\usepackage[UTF8]{ctex}
\usepackage{graphicx}
\usepackage{amsmath}

\title{Shell脚本的介绍和编写}

\author{鲁衍坤 \\ 信计2101 3210105173}
\date{June 28, 2022}

\begin{document}

\maketitle

某位大佬曾经说过,Shell归愚蠢的人类使用。但是,Shell实际上功能相当便捷,甚至可以在Shell里写代码,更确切地说,脚本。本文旨在分享笔者最近所学习到的Shell脚本的基础知识和一点点实际应用。话不多说,直接进入正题。

\section{预备知识与测试}

\begin{verbatim}
$ ./test abc
\end{verbatim}

这是一句十分简单的Shell命令,其作用是运行当前目录下的test文件(如果它可以运行的话)。
(获取运行该文件的权限也很简单,只需要一句\verb|$ chmod a+x test|即可。)

不过,我们发现命令可以不单单只有命令,我们可以为即将要执行的文件传入一些参数,具体来说,就是上述命令中的不明意义字符串\verb|abc|。

并且,Shell自带的一些命令,如\verb|read|或者\verb|echo|等,都是支持输入输出等操作的。除此之外,Shell也允许自己创建变量,这些都给了我们使用Shell命令编写脚本的可能。

\textbf{这一章节接下来的shell脚本内容均写与该pdf文件同时上交的文件ShellCode一致,并且测试结果均出自于ShellCode文件的测试结果。}

\subsection{环境变量}

当Shell脚本启动时,一些变量会根据配置环境进行初始化。这些变量通常都是大写的,以区别于脚本中的用户定义变量。创建的变量取决于电脑上的个人配置。一些基础的环境变量如下表所示:

\begin{tabular}{|l|p{18em}|}
  \hline
  环境变量 & 描述\\
  \cline{1-2}
  \$HOME & 用户主要文件夹(home)的地址。\\
  \$PATH & 用于搜索命令的以冒号分割的地址。\\
  \$PS1 & 命令提示符,通常为\$,但在bash中可以使用更复杂的值;例如,字符串\verb|[\u@\h\W]$|是常用默认值,告诉您用户、计算机名称和当前目录,并提供\$提示。\\
  \$PS2 & 辅助提示,用于添加其他输入;通常为>。\\
  \$IFS & 输入字段分隔符。用于shell读取输入时,通常是space、tab和换行符。\\
  \$0 & Shell脚本的名称。\\
  \$\# & 传递的参数数量。\\
  \$\$ & Shell脚本的进程ID。\\
  \hline
\end{tabular}

这些环境变量都是脚本中随时可以调用的常数。利用它们能做到许多操作。

\subsection{输入、输出}

先看一段简单的脚本代码:

\begin{verbatim}
echo "Please enter a new greeting"
read salutation
echo $salutation
\end{verbatim}

它们运行的结果是

\begin{tabular}{|l|}
  \hline
Please enter a new greeting\\
\textbf{helloworld}\\
helloworld\\
\hline
\end{tabular}

所以,echo在shell中有输出的功能,而read则可以读入一个字符串。需要注意的是,echo输出后会自动换行,而read读入的字符串也只读到空格为止。

当然,shell也支持自动创建变量,也就是脚本中的\verb|salutation|(通常为字符串)。

\subsection{条件、循环}

要想完成相对而言更加复杂和更加实际的问题,条件结构和循环结构必不可少。当然,shell脚本支持这些操作。

首先是条件结构:

\begin{verbatim}
echo "plz input a sentence"
read str
if [ $str = "GenShin" ]
then
   echo "woc!yuan!"
elif [ "$str" = "Arknight" ]
then
   echo "woc!zhou!"
else
   echo "not erciyuan gamer"
   exit 1
fi
\end{verbatim}

分别运行三次,不同的输入对应的结果是

\begin{tabular}{|l|}
  \hline
\textbf{GenShin}\\
woc!yuan!\\
\\
\textbf{Arknight}\\
woc!zhou!\\
\\
\textbf{Terraria}\\
not erciyuan gamer\\
\hline
\end{tabular}

这样的条件语句与python中的判断语句写法类似,用\verb|then-fi|代替了大括号。

需要注意的是,\verb|$str|和\verb|"$str"|在判断时并无影响,加双引号只是为了统一格式。

然后是循环结构:

\begin{verbatim}
for foo in bar fud 43
do
   echo $foo
done
\end{verbatim}

运行的结果是

\begin{tabular}{|l|}
  \hline
bar\\
fud\\
43\\
\hline
\end{tabular}

当然,也有while形式的写法:
\begin{verbatim}
echo "Enter My Username"
read trythis
while [ "$trythis" != "FallCloud" ]
do
   echo "Sorry, try again"
   read trythis
done
\end{verbatim}

运行的结果是

\begin{tabular}{|l|}
  \hline
Enter My Username\\
\textbf{idiot}\\
Sorry, try again\\
\textbf{noob}\\
Sorry,try again\\
\textbf{FallCloud}\\
\hline
\end{tabular}

两种循环结构的写法也都和python类似(但是正经人谁学python)。

\subsection{函数}

是的,在Shell里,你甚至可以写一个脚本函数。

就像这样:

\begin{verbatim}
foo() {
   echo "Function foo is executing"
}
echo "script starting"
foo
echo "script ended"
\end{verbatim}

运行结果如下:

\begin{tabular}{|l|}
\hline
  script starting\\
Function foo is executing\\
script ended\\
\hline
\end{tabular}

当然,函数也可以有返回值。不过需要注意的是,\verb|return 0|代表返回\verb|true|,而\verb|return 1|代表返回\verb|false|。笔者很疑惑这样设置的理由(恼

  \section{编写脚本与结果}
  
既然shell支持这么多功能,我就用粗浅的理解写下了一个功能简单的脚本。\textbf{读者应当可以在与本文本相同目录下找到一个名为ShellScript的文件},这个文件即为这里所指的脚本。

其源码如下:
\begin{verbatim}
#!/bin/bash

check() {
  for num in 1 2 3 4 5
  do
    if [ "$1" = "$num" ]
    then
      echo "This is $num"
      return 0
    else
      echo "This is NOT $num"
    fi
  done
  
  return 1
}

echo "Original parameters are $*"
echo "Your first number is $1"
echo "now enter a number again"
read n
while true
do
  if check "$n" 
  then
    echo "easy!I'm good at mathmatics"
    echo "try another number"
  else
    echo "TOO BIG!!!"
    echo "That's a bad number.I can not do the conut."
    exit 1
  fi
  read n
done

exit 0
\end{verbatim}

该代码的作用为:首先获取命令后全部的参数序列和命令后的第一个参数并输出,然后要求用户输入数字,如果输入的内容是1、2、3、4、5的其中一个,那么就会从一开始数到输入的数字。如果输入的是其他内容,则输出错误信息然后退出。

测试结果如下:

\begin{tabular}{|l|}
  \hline
\textbf{./ShellScript 7 8 9}\\
Original parameters are 7 8 9\\
Your first number is 7\\
now enter a number again\\
\textbf{3}\\
This is NOT 1\\
This is NOT 2\\
This is 3\\
easy!I'm good at mathmatics\\
try another number\\
\textbf{0x3f}\\
This is NOT 1\\
This is NOT 2\\
This is NOT 3\\
This is NOT 4\\
This is NOT 5\\
TOO BIG!!!\\
That's a bad number.I can not do the count.\\
\hline
\end{tabular}

这个代码中使用到了上文所介绍到的所有功能。当然,Shell所能做到的操作远不止于此,本文只做简单分享。

\end{document}
