\documentclass[UTF8]{article}

\usepackage[UTF8]{ctex}
\usepackage{graphicx}
\usepackage{amsmath}
\usepackage{amssymb}
\usepackage{subfigure}

\title{roots.c功能简介}

\author{鲁衍坤 \\ 信计2101 3210105173}
\date{July 2, 2022}

\bibliographystyle{plain}

\begin{document}

\maketitle

本文是gsl自带的一个example源码功能的简介。源码文件为\verb|../src/roots.c|。

其功能为:使用布伦特法(brent's method)求解$\sqrt{5}$的近似解,并输出中间过程。其输出如下所示:

\begin{verbatim}
using brent method
 iter [    lower,     upper]      root        err  err(est)
    1 [1.0000000, 5.0000000] 1.0000000 -1.2360680 4.0000000
    2 [1.0000000, 3.0000000] 3.0000000 +0.7639320 2.0000000
    3 [2.0000000, 3.0000000] 2.0000000 -0.2360680 1.0000000
    4 [2.2000000, 3.0000000] 2.2000000 -0.0360680 0.8000000
    5 [2.2000000, 2.2366300] 2.2366300 +0.0005621 0.0366300
Converged:
    6 [2.2360634, 2.2366300] 2.2360634 -0.0000046 0.0005666
\end{verbatim}

记录了迭代次数、下限、上限、近似解、近似解与实际解的差值、上下限的差值。
其中第六次迭代达到了要求的精度,程序结束并输出最终近似解。

\bibliography{quote}

\end{document}
